%%% START A [ 829 words + 29 for the last sentence ]

The activity in the system comprised of the basal ganglia, cerebellum and motor cortex plus the brain stem and spinal cord is perhaps the closest the mammalian brain comes to writing and executing programs. Generally referred to as "motor programs", they control all of your voluntary movements and, in humans, they play an important role in language.

The right panel in Figure~{\urlh{#fig_White_Matter_Tracts_Long_Distance}{\ref{fig_tracts}}} shows the white matter reciprocal connections between the frontal cortex and the cerebellum that are believed to facilitate higher-order cognitive functions. Curiously, it is possible to lead a relatively normal life even if you were born without a cerebellum, difficulty speaking and walking being the most obvious deficits.

The distribution of cell types and neural circuitry of the cerebellum is reminiscent of the hippocampal formation and there are several detailed models of the cerebellum~\cite{MarrJoP-69,AlbusMB-71,ItoCEREBELLAR-CORTEX-18} that have inspired useful machine learning techniques~\cite{Albus75}, and yet discoveries in the last decade have challenged prevailing opinions.

It was believed that all communication between the basal ganglia and cerebellum was indirectly enabled via the cerebral cortex, but evidence now supports the existence of subcortical connections between the two suggesting that the basal ganglia, the cerebellum and the cerebral cortex form an integrated network~\cite{BostanandStrickNATURE-REVIEWS-NEUROSCIENCE-18}.

These new discoveries will likely have important ramifications for our understanding of these critical systems that will lead to new algorithmic insights that parallel those fueled by our study of the hippocampal formation Box~\colorred{D}. Here we focus on what the basal ganglia, the cerebellum and the cerebral cortex tell us about creating, selecting and coordinating motor programs.

The brain derives much of its utility from exploiting distributed representations and parallel processing. Even so, in big brains it is often useful bring representations from distant parts of the brain together and necessary to perform some computations serially with the results from one computation feeding into another. The human brain has evolved machinery that makes it possible for us to do both by making better use of existing memory systems and adapting circuits optimized for movement to communicate, plan and perform abstract reasoning.

The human brain makes extensive use of topographically organized representations, often contructing multiple maps with same topographic organization that can be aligned with one another to construct more abstract representation that retain the locality relationships of their constituent maps~\cite{WandelletalNEURON-07,WandelletalPTRS-B-05}. This is a subtle point and a generally underappreciated fact about organisms with a central nervous system responsible for coordinating behavior distributed across their peripheral nervous system. 

Several of the subcortical circuits we've discussed including the basal ganglia and hippocampus have access to sensorimotor areas of the cerebral cortex by way of the thalamus and the striatum, the latter being part of the basal ganglia. The thalamus consists of a set of nuclei that map specific subcortical inputs to the cortex and receive feedback from the same cortical areas. The striatum assists in coordinating cognitive functions, including both motor and action planning.

Much of the information moving around in the cortex is structured in the form of maps that align with the topography of the sensory and motor systems of the body, retinotopy of the visual cortex being one example. The striatum's distinctive striated appearance is due to its arrangement of specialized circuits called {\it{stripes}} that enable the basal ganglia to select and transfer information to distant locations in the cortex that have similarly striated and functioning circuits, preserving essential topographical features in the process~\cite{BarbasandGarcia-CabezasCOiN-16,LewisetalJNC-02}.

Each stripe is composed of columnar-shaped circuits called {\it{minicolumns}} that encode patterns of coordinated activity originating elsewhere in the cortex so it can brought together in one place for processing. Each stripe can be updated independently allowing the basal ganglia a great deal of discretion in creating a context for initiating subsequent computations at remote locations in the frontal cortex. These patterns of activity can be stored indefinitely providing a working memory system that supports a simple yet powerful method for binding variables and composing their values~\cite{OReillyetalCCN-12}.

Stripes are grouped in {\it{clusters}} that can reinforce or inhibit one another and clusters map to other clusters often by way of white-matter tracts that connect distant sensory and motor areas. Information is transferred preserving its topographic structure so processed information resulting from motion planning or other cognitive functions can be mapped back to its origin to support learning or stimulate muscle activity.

The activity in stripe clusters can be maintained or updated individually allowing for sustained or iterative processing and providing the basis for working memory. Information in multiple clusters can be combined to support a simple form of variable binding. The basal ganglia and prefrontal cortex can influence what information is transferred but the preserved alignment of the stripes within clusters dictates the origin of the information.

At the risk of oversimplifying, stripes are memory locations and white matter tracts are high-speed buses that tranfer data to and from the processor arithmetic and logic unit.

%%% STOP A 
