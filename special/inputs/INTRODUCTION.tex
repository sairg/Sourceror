%%% .File: /inputs/INTRODUCTION.tex

%%% %%%%%%%%%%%%%%%%%%%%%%%%%%%%%%%%%%%%%%%%%%%%%%%%%%%%%%%%%%%%%%%%%%%%%%%%%%%%

\section{Introduction}

%%% %%%%%%%%%%%%%%%%%%%%%%%%%%%%%%%%%%%%%%%%%%%%%%%%%%%%%%%%%%%%%%%%%%%%%%%%%%%%

%%% \begin{quotation}
%%% %
%%%   We are interested in designing neural network architectures that leverage what is known about biological information processing to solve complex real-world problems. To focus our efforts, we have set out to design end-to-end systems that assist human programmers in writing, debugging and modifying software. We benefit considerably from working closely with scientists from diverse subdisciplines of neuroscience to seek solutions to specific problems and identify additional problems we may have overlooked. The following section explains why this commingling of people, ideas and technologies is so valuable to us in the pursuit of our objectives.
%%% %
%%% \end{quotation}

Artificial neural networks support distributed computations in which concepts are represented as patterns of activity in the units that comprise the network layers, and inference is carried out by propagating activation levels between layers weighted by learned connection weights.  Artificial neural networks provide a type of fast, flexible computing well suited to handling ambiguity of the sort we routinely encounter in real-world environments, and, by doing so, they complement traditional symbolic computing technologies.

Engineers frequently borrow ideas from nature and generally find it more practical to translate these ideas into current technology rather than attempt to reproduce nature's solutions in detail. Indeed, the basic idea of artificial neural networks has been implemented multiple times using different technologies in order to approximate the connectivity patterns and signal transmission characteristics of real neural circuits while largely ignoring the physiology of real neurons in their implementation. 

%%% ~\cite{McClellandandCleeremansCONNECTIONISM-09}.

%%% Modern artificial neural networks have come a long way in the last twenty years and many of the innovations were inspired by what we have learned from the various sub-disciplines of neuroscience. In this paper, we describe our work in designing digital assistants by drawing upon the many insights we have gathered from collaborating with cellular, systems and cognitive neuroscientists. We also survey additional ideas drawn from current research that we are exploring in an attempt to address more ambitious challenges.

%%% The human brain is anything but a blank slate. Neither is it a homogeneous network of neurons all of which perform the same function. It is an incredible self-assembling computing device that emerges out of a collection of embryonic stem cells having the potential to become any cell in the adult brain. These cells have the ability to sense and respond to their environment. They migrate and self organize to construct complex multi-cellular machines and communicate using chemical, electrical and genetic pathways~\cite{RichterandGjorgjievaBIORXIV-17}.

%%% During its first few years the infant brain requires 60\% of the body's metabolic budget. The number of neurons and synapses increases to nearly double that of the adult brain before undergoing extensive pruning during adolescence. Temporary scaffolding is erected to move neurons to appropriate locations and neural processes are guided by chemotaxis to make tentative connections with other neurons. While heightened through adolescence, the same sort of activity continues throughout life and is crucial to learning and memory.

The human brain supports a wide array of learning and memory systems. Some we have begun to understand functionally and behaviorally, others we can only hypothesize must exist, and still others about which we haven't a clue. Just knowing {\it{that}} the brain supports a particular capability can serve as an important clue in engineering complex AI systems. Knowing {\it{how}} can lead to an innovative design, enhanced performance and extended competence. In particular, knowing something about how specific biological circuits relate to behavior helps in designing novel network architectures.

We are interested in designing neural network architectures that leverage what is known about biological information processing to solve complex real-world problems. To focus our efforts, we have set out to design end-to-end systems that assist human programmers in writing, debugging and modifying software. We benefit considerably from working closely with scientists from diverse subdisciplines of neuroscience to seek solutions to specific problems and identify additional problems we may have overlooked. The following section explains why this commingling of people, ideas and technologies is so valuable to us in pursuit of our objectives.

%%% Knowing how the brain implements a given capability seldom leads directly to an implementation in silicon. A biological realization might rely on diffuse signaling with neuromodulators~\cite{BrezinaPTRS-B-10}, multiplexing by synchronous spiking~\cite{LankaranyetalPNAS-19}, DNA methylation for memory formation~\cite{OliveiraCSH-16}, neurogenesis in pattern separation~\cite{AimoneetalNEURON-09,BurghardtetalHIPPOCAMPUS-12}, synchronized oscillations for memory consolidation during sleep~\cite{JooandFrankNATURE-REVIEWS-NEUROSCIENCE-18,OuchietalCRN-17,BuzsakiHIPPOCAMPUS-15}, or any of the myriad mechanisms the brain employs to perform its diverse functions. Understanding the biology well enough to convert such knowledge into practical algorithms requires a collaborative effort.

%%% It's not just these specialized mechanisms selected for their reproductive edge that are valuable to the engineer designing AI systems; the highly conserved circuitry of the vertebrate brain offers a wealth of architectural detail aligned with cognitive function. While generally not extending to the level of individual neurons, much of this structure is highly stereotyped across many species allowing us to draw on a much wider range of research in trying to emulate human cognitive functions\cite{PortuguesetalNEURON-14}. 

%%% This is especially evident in the primary sensorimotor cortex where the number, arrangement and topography of the areas that comprise the first stages of processing are remarkably stereotypical~\cite{WandelletalNEURON-07,WandelletalPTRS-B-05}. In terms of more general localized functions, action selection~\cite{HokeetalICB-17}, social behavior~\cite{ChenandHongNEURON-18}, mathematical and related analytical reasoning~\cite{AmalricandDehaenePNAS-16,MenonPBR-16} and even the signature basis for abstraction and hierarchical reasoning~\cite{BadreandFrankCEREBRAL-CORTEX-12,KoechlinandJubaultNEURON-06} all provide useful hints about how to organize complex artificial neural network architectures.

%%% Knowing something about how circuit activity relates to behavior also helps in thinking about suitable network architectures and the connections between subnetworks in deep networks including recurrent connections. There is evidence to suggest that circuits occurring early in the ventral visual stream code for object-selective features and exhibit large-scale organization characterized by the high-level properties of both animacy and object size~\cite{LongetalPNAS-18,KonkleandCaramazzaJoN-13}. The point being that if an organism learns a feature then it serves some purpose relating to its survival and reproductive success.

%%% The visual cortex is far and away the most well studied part of the mammalian brain\footnote{%
%%% %
%%%   The primary visual cortex is the most studied visual area in the brain. In mammals, it is located in the posterior pole of the occipital lobe and is the simplest, earliest cortical visual area. It is highly specialized for processing information about static and moving objects and is excellent in pattern recognition. ({\urlh{https://en.wikipedia.org/wiki/Visual_cortex}{SOURCE}})}.
%%% %
%%% It is much more challenging to identify and understand features and functions that are further from sensory-motor periphery, including systems responsible for memory, decision making and executive control.  Fortunately, cellular, cognitive and systems neuroscience, developmental psychology and evolutionary biology are just a few of disciplines we can call on in looking for architectural and algorithmic guidance in designing intelligent systems.

%%% %%%%%%%%%%%%%%%%%%%%%%%%%%%%%%%%%%%%%%%%%%%%%%%%%%%%%%%%%%%%%%%%%%%%%%%%%%%%
