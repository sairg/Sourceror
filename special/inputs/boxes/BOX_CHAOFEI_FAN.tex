%%% File: ./inputs/boxes/BOX_CHAOFEI_FAN.tex

\begin{center}
  %%% \begin{tcolorbox}[sharp corners=all,coltitle=black,colbacktitle=white,
  \begin{tcolorbox}[breakable,sharp corners=all,coltitle=black,colbacktitle=white,
    width=\textwidth,boxsep=5pt,left=5pt,right=5pt,
    title={\textbf{Box A: Pattern Separation, Completion and Integration}}]

    %%% width=\textwidth,boxsep=5pt,left=5pt,right=5pt,hypertarget={box_patterns},

    %%% Pattern separation is defined as the process by which overlapping or similar inputs
    %%% (representations) are transformed into less similar outputs [...] pattern completion
    %%% is defined as the reconstruction of complete stored representations from partial 
    %%% inputs that are part of the stored representation (Colgin et al., 2008; Wilson, 2009). 

~~~~{\it{Pattern separation}} reduces the similarity between input patterns of activity by orthogonalizing inputs to minimize interference between patterns and increase hippocampal storage capacity~\cite{KesnerandRollsNBR-15}. Pattern separation involves primarily dentate gyrus (DG) and hippocampal CA3. The DG maps input from entorhinal cortex (EHC) to a much larger and sparsely active granule cells (GCs) population. In rats, the number of neurons in the DG exceeds that in EHC by about 5:1~\cite{DrewetalLEARNING-MEMORY-13}. This expansion coding with strong inhibitory interneurons and a competitive learning rule can greatly reduce the overlap between inputs. The DG connects to CA3 mainly through {\it{mossy fibers}} that reliably activate CA3 pyramidal neurons and sustain activation for tens of seconds~\cite{VyletaetalELIFE-16}. Each CA3 neuron receives a small number of these connections from DG so the degree of sparsity is maintained~\cite{KesnerandRollsNBR-15}.

~~~~{\it{Pattern completion}} reconstructs the complete stored pattern given a partial input. Each pyramidal neuron in CA3 receives a large number of synapses from other pyramidal cells forming a recurrent network that serves as an autoassociative memory for pattern completion~\cite{KesnerandRollsNBR-15}. During learning, recurrent connections between active CA3 neurons are strengthened and later when neurons encoding part of an episode are reactivated, they recurrently activate other connected cells to reconstruct the original episode. Basket cells in CA3 form inhibitory synapses to pyramidal cells to dampen excitatory responses thereby emphasizing key features~\cite{NeunuebelandKnierimNEURON-14}. 

~~~~Pattern completion provides access to relevant experience to support decision making in novel situations, and while pattern separation helps downstream discrimination, perfectly orthogonal representations are not ideal in the case we want events that occurred close together to have similar representations. In this case, {\it{pattern integration}} represents related experiences as overlapping populations. There are a number of neural mechanisms suggested to support pattern integration in the hippocampus. We consider two here, the first of which involves {\it{neurogenesis}}. 

~~~~There is evidence that hundreds of new GCs are added to an adult human hippocampus everyday~\cite{SpaldingetalCELL-13}, and stronger evidence suggests that thousands of new GCs are added to rodent’s hippocampus, though not all survive~\cite{KitabatakeetalNCNM-07}. Unlike mature GCs that fire sparsely, immature GCs are more active and have lower threshold for induction of long-term potentiation~\cite{AimoneetalNEURON-09,GeetalNATURE-06,Schmidt-HieberetalNATURE-04}. Aimone~\etal{}~\cite{AimoneetalNEURON-09} posit that a population of hyperactive young GCs could collectively encode events close in time to decrease pattern separation in DG. Others hypothesize that neurogenesis may increase storage capacity by protecting old GCs from new information~\cite{BeckerHIPPOCAMPUS-05,WiskottetalHIPPOCAMPUS-06} or that young active GCs could improve the resolution of memory content~\cite{AimoneetalNEURON-11}. 

~~~~Alternatively, pattern integration might be enabled by recurrent connections involving the hippocampus and neocortex. Recurrent connections in the hippocampus, mainly in CA3 region, can replay an entire episode given a part of it. The replayed episode is backprojected to the neocortex through EHC, that can then recirculate the replayed episode as input to hippocampus to trigger replay of another episode that has overlapping elements with previous one. Kumaran~\etal{}~\cite{KumaranetalTiCS-16} propose that this kind of replay between hippocampus and neocortical regions can combine representations of elements that seldom occur together but appear in similar contexts. In addition to integrating experiences with shared elements, backprojection to the medial prefrontal cortex (mPFC) may bias hippocampus to reactivate experiences that are more behaviorally relevant~\cite{SchlichtingandPrestonCOiBS} {\emdash{}} see {\urlh{box_memories}{Box~\colorred{B}}} for more on behavioral relevance. The concurrent presentation of these memories in mPFC may further improve the learning of abstraction relations across episodes.

  \end{tcolorbox}
\end{center}
